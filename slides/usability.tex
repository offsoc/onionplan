% Options for packages loaded elsewhere
\PassOptionsToPackage{unicode}{hyperref}
\PassOptionsToPackage{hyphens}{url}
%
\documentclass[
  ignorenonframetext,
]{beamer}
\usepackage{pgfpages}
\setbeamertemplate{caption}[numbered]
\setbeamertemplate{caption label separator}{: }
\setbeamercolor{caption name}{fg=normal text.fg}
\beamertemplatenavigationsymbolsempty
% Prevent slide breaks in the middle of a paragraph
\widowpenalties 1 10000
\raggedbottom
\setbeamertemplate{part page}{
  \centering
  \begin{beamercolorbox}[sep=16pt,center]{part title}
    \usebeamerfont{part title}\insertpart\par
  \end{beamercolorbox}
}
\setbeamertemplate{section page}{
  \centering
  \begin{beamercolorbox}[sep=12pt,center]{part title}
    \usebeamerfont{section title}\insertsection\par
  \end{beamercolorbox}
}
\setbeamertemplate{subsection page}{
  \centering
  \begin{beamercolorbox}[sep=8pt,center]{part title}
    \usebeamerfont{subsection title}\insertsubsection\par
  \end{beamercolorbox}
}
\AtBeginPart{
  \frame{\partpage}
}
\AtBeginSection{
  \ifbibliography
  \else
    \frame{\sectionpage}
  \fi
}
\AtBeginSubsection{
  \frame{\subsectionpage}
}
\usepackage{lmodern}
\usepackage{amssymb,amsmath}
\usepackage{ifxetex,ifluatex}
\ifnum 0\ifxetex 1\fi\ifluatex 1\fi=0 % if pdftex
  \usepackage[T1]{fontenc}
  \usepackage[utf8]{inputenc}
  \usepackage{textcomp} % provide euro and other symbols
\else % if luatex or xetex
  \usepackage{unicode-math}
  \defaultfontfeatures{Scale=MatchLowercase}
  \defaultfontfeatures[\rmfamily]{Ligatures=TeX,Scale=1}
\fi
% Use upquote if available, for straight quotes in verbatim environments
\IfFileExists{upquote.sty}{\usepackage{upquote}}{}
\IfFileExists{microtype.sty}{% use microtype if available
  \usepackage[]{microtype}
  \UseMicrotypeSet[protrusion]{basicmath} % disable protrusion for tt fonts
}{}
\makeatletter
\@ifundefined{KOMAClassName}{% if non-KOMA class
  \IfFileExists{parskip.sty}{%
    \usepackage{parskip}
  }{% else
    \setlength{\parindent}{0pt}
    \setlength{\parskip}{6pt plus 2pt minus 1pt}}
}{% if KOMA class
  \KOMAoptions{parskip=half}}
\makeatother
\usepackage{xcolor}
\IfFileExists{xurl.sty}{\usepackage{xurl}}{} % add URL line breaks if available
\IfFileExists{bookmark.sty}{\usepackage{bookmark}}{\usepackage{hyperref}}
\hypersetup{
  hidelinks,
  pdfcreator={LaTeX via pandoc}}
\urlstyle{same} % disable monospaced font for URLs
\newif\ifbibliography
\setlength{\emergencystretch}{3em} % prevent overfull lines
\providecommand{\tightlist}{%
  \setlength{\itemsep}{0pt}\setlength{\parskip}{0pt}}
\setcounter{secnumdepth}{-\maxdimen} % remove section numbering

\author{}
\date{}

\begin{document}

\begin{frame}{Onion Plan: Usability Roadmap Proposal - 2022.Q4}
\protect\hypertarget{onion-plan-usability-roadmap-proposal---2022.q4}{}
\end{frame}

\begin{frame}{What}
\protect\hypertarget{what}{}
Onion Plan: an ongoing strategy to increase the adoption an enhance the
usability of
\href{https://community.torproject.org/onion-services/}{Onion Services}.
\end{frame}

\begin{frame}{Have you ever?}
\protect\hypertarget{have-you-ever}{}
\textbf{\emph{Have you ever considered that we work with one of the
coolest technologies?}}

\textbf{\emph{And that our job consists in making it even cooler?}}
\end{frame}

\begin{frame}{But beware!}
\protect\hypertarget{but-beware}{}
\begin{itemize}
\tightlist
\item
  Here follows a non-orthodox strategy to improve Onion Services UX.
\item
  It's meant to balance between the present and urgent user needs and
  the wish to have fully distributed Onion Names in the future.
\item
  It's an incremental roadmap, focusing on what's more feasible to do
  first instead of targeting in systems that still need to mature.
\end{itemize}
\end{frame}

\begin{frame}{And now imagine}
\protect\hypertarget{and-now-imagine}{}
\textbf{Imagine} a \textbf{communication technology} that has:

\begin{enumerate}
\tightlist
\item
  Built-in resistance against surveillance, censorship and denial of
  service.
\item
  Built-in end-to-end encryption.
\item
  An address space bigger than IPv6 and without allocation authority.
\item
  Support for multiple, pluggable naming systems.
\item
  And that also works as an anonymization layer.
\end{enumerate}
\end{frame}

\begin{frame}{Enhanced Onion Services}
\protect\hypertarget{enhanced-onion-services}{}
We may call this technology \textbf{Enhanced Onion Services}!

Note a shift in how the technology is presented: instead of first
stating that it's an anonymization technology, now the focus is
\emph{protection against surveillance, censorship and DoS} with
\emph{built-in anonymity in the Onion Service protocol}. This can make
it easier to showcase the techology and attract potential funders.
\end{frame}

\begin{frame}{What's still missing}
\protect\hypertarget{whats-still-missing}{}
\begin{enumerate}
\tightlist
\item
  Built-in DoS resistance.
\item
  Pluggable discoverability (multiple naming systems).
\item
  Many other enhancements in usability and tooling.
\end{enumerate}
\end{frame}

\begin{frame}{Tracks}
\protect\hypertarget{tracks}{}
This plan is split into the following roadmap tracks:

\begin{enumerate}
\tightlist
\item
  Health: DoS protections, performance improvements etc.
\item
  \textbf{Usability: Onion Names, Tor Browser improvements etc.}
\item
  Tooling: Onionbalance, Onionprobe, Oniongroove etc.
\item
  Outreach: documentation, support, usage/adoption campaigns etc.
\end{enumerate}
\end{frame}

\begin{frame}{Usability Roadmap}
\protect\hypertarget{usability-roadmap}{}
\begin{itemize}
\tightlist
\item
  \textbf{Focus:} \textbf{\emph{human-friendly}} names for Onion
  Services with \textbf{\emph{HTTPS support}}.
\item
  \textbf{Goal:} \textbf{\emph{coexistence}} between different methods
  and \textbf{\emph{opportunistic discovery}}.
\item
  \textbf{Characteristics}: \textbf{\emph{pragmatic, modular,
  incremental, backwards compatible, future-proof and risk-minimizing}}
  phases.
\end{itemize}
\end{frame}

\begin{frame}[fragile]{Phases}
\protect\hypertarget{phases}{}
\begin{itemize}
\tightlist
\item
  Phase 1: accessing URLs like \texttt{https://torproject.org} directly
  using Onion Services and HTTPs!
\item
  Phase 2: \emph{opportunistic discovery} of .onion addresses (increased
  censorship resistance).
\item
  Phase 3: bringing ``pure'' Onion Names into Tor.
\end{itemize}
\end{frame}

\begin{frame}{Phase 0}
\protect\hypertarget{phase-0}{}
We're at Phase 0, but not starting from zero! :)

\begin{itemize}
\tightlist
\item
  We have Onion Services v3!
\item
  We have accumulated lots of discussions, proposals and analisys.
\end{itemize}
\end{frame}

\begin{frame}[fragile]{Phase 1}
\protect\hypertarget{phase-1}{}
Objective: accessing URLs like \texttt{https://torproject.org} directly
using Onion Services and HTTPs!

That means:

\begin{enumerate}
\item
  It \emph{can be transparent}, either by always preferring the Onion
  Service or using it automatically if the regular site is blocked.
\item
  Users will not need to know the actual Onion Service address!
\item
  Can work for all clients and not only Tor Browser.
\end{enumerate}
\end{frame}

\begin{frame}{Tor Browser}
\protect\hypertarget{tor-browser}{}
For Tor Browser, it can be possible to have special interface indicators
to inform users:

\begin{itemize}
\tightlist
\item
  How the connection to the site is happening.
\item
  Which available connection options exists for the site (regular or via
  .onion).
\end{itemize}
\end{frame}

\begin{frame}[fragile]{But how it would work?}
\protect\hypertarget{but-how-it-would-work}{}
\begin{enumerate}
\item
  Transparent resolution of \texttt{torproject.org} into
  \texttt{2gzyxa5ihm7nsggfxnu52rck2vv4rvmdlkiu3zzui5du4xyclen53wid.onion}
  using DNS via Tor.
\item
  Use the \emph{existing} HTTPs certificate for \texttt{torproject.org},
  with no need to have
  \texttt{2gzyxa5ihm7nsggfxnu52rck2vv4rvmdlkiu3zzui5du4xyclen53wid.onion}
  in the certificate!
\item
  TLS SNI connection to
  \texttt{https://2gzyxa5ihm7nsggfxnu52rck2vv4rvmdlkiu3zzui5du4xyclen53wid.onion}
  using \texttt{torproject.org} as the server name.
\end{enumerate}
\end{frame}

\begin{frame}{What it needs to work?}
\protect\hypertarget{what-it-needs-to-work}{}
\begin{enumerate}
\item
  Transparent resolution:

  \begin{itemize}
  \tightlist
  \item
    \href{https://gitlab.torproject.org/tpo/core/torspec/-/blob/main/proposals/279-naming-layer-api.txt}{Proposal
    279} (2016) - specs for a Tor Name System API: review and
    implementation.
  \item
    Define a way to securely add Onion Service addresses entries into
    the DNS.
  \item
    Write a Tor NS API plugin that securely maps regular domains into
    Onion Services.
  \item
    Minimum UX changes in the Tor Browser.
  \end{itemize}
\item
  HTTPS Certificates: Already supported!
\item
  TLS SNI: Already supported!
\end{enumerate}
\end{frame}

\begin{frame}[fragile]{Phase 2}
\protect\hypertarget{phase-2}{}
Objective: \emph{increase the censorship resistance} of accessing URLs
like \texttt{https://torproject.org} directly using Onion Services and
HTTPs!

That means:

\begin{itemize}
\item
  Implementing \emph{opportunistic discovery} of Onion Service addresses
  by having an additional method to get the .onion address for
  \texttt{torproject.org}.
\item
  In this phase, a \href{https://www.sauteed-onions.org}{Sauteed Onions}
  Tor NS plugin will be created.
\end{itemize}
\end{frame}

\begin{frame}[fragile]{Phase 3}
\protect\hypertarget{phase-3}{}
Objective: bring ``pure''/``real'' Onion Names into Tor.

That means:

\begin{itemize}
\tightlist
\item
  Transparent access to \texttt{http://somesite.some.onion}.
\item
  Having techincal and governance specs to decide which Onion Names are
  officially accepted.
\item
  Allocating a namespace (at \texttt{.onion}?) to each proposal.
\item
  Optionally shipping the implementation into a bundle for distribution.
\end{itemize}
\end{frame}

\begin{frame}{Technical details}
\protect\hypertarget{technical-details}{}
\begin{itemize}
\tightlist
\item
  Proposal 279 overview.
\item
  DNS, TLS SNI and .onion proof of concept.
\end{itemize}
\end{frame}

\begin{frame}[fragile]{Proposal 279 (2016)}
\protect\hypertarget{proposal-279-2016}{}
\begin{quote}
{[}\ldots{]} a modular Name System API (NSA) that allows developers to
integrate their own name systems in Tor. {[}\ldots{]} It should be
flexible enough to accommodate all sorts of name systems

{[}\ldots{]} Tor asks the name system to perform name queries, and
receives the query results. {[}\ldots{]} It aims to be portable and easy
to implement.

---
https://gitlab.torproject.org/tpo/core/torspec/-/blob/main/proposals/279-naming-layer-api.txt
\end{quote}

\begin{verbatim}
# New torrc(5) config
OnionNamePlugin 0 .hosts.onion    /usr/local/bin/local-hosts-file
OnionNamePlugin 1 .zkey.onion     /usr/local/bin/gns-tor-wrapper
OnionNamePlugin 2 .bit.onion      /usr/local/bin/namecoin-tor-wrapper
OnionNamePlugin 3 .scallion.onion /usr/local/bin/community-hosts-file
\end{verbatim}
\end{frame}

\begin{frame}{TorNS (2017-2019)}
\protect\hypertarget{torns-2017-2019}{}
\begin{itemize}
\tightlist
\item
  Tor NS API proof of concept.
\item
  https://github.com/meejah/torns
\end{itemize}
\end{frame}

\begin{frame}[fragile]{What if\ldots?}
\protect\hypertarget{what-if}{}
\begin{verbatim}
# New torrc(5) config
OnionNamePlugin  0 .some.onion /usr/bin/some-onion-resolver # Phase 3
OnionNamePlugin 98 *           /usr/bin/dns-to-onion-resolver   # Phase 1
OnionNamePlugin 99 *           /usr/bin/sauteed-onion-resolver  # Phase 2
\end{verbatim}
\end{frame}

\begin{frame}[fragile]{Which means}
\protect\hypertarget{which-means}{}
\begin{enumerate}
\tightlist
\item
  In Phase 1, the DNS-based address translation is implemented.
\item
  In Phase 2, the Sauteed Onions address translation is implemented.
\item
  In Phase 3, ``pure'' Onion Name plugins can be officially included.
\item
  Matching will happen from the specific (like \texttt{.some.onion}) to
  the general (\texttt{*}).
\item
  For non-.onion TLDs, priority will be from the DNS to the Sauteed
  Onion (or other fancier methods).
\end{enumerate}
\end{frame}

\begin{frame}{DNS, TLS SNI and .onion: proof of concept}
\protect\hypertarget{dns-tls-sni-and-.onion-proof-of-concept}{}
\begin{itemize}
\tightlist
\item
  Using an existing site: https://autodefesa.org
\item
  Using it's existing Onion Service:
  autodefcecpx2mut5medmyjxjg2wb6lwkbt3enl74frthemyoyclpiad.onion
\end{itemize}
\end{frame}

\begin{frame}{Today's behavior}
\protect\hypertarget{todays-behavior}{}
\begin{itemize}
\tightlist
\item
  Accessing:
  https://autodefcecpx2mut5medmyjxjg2wb6lwkbt3enl74frthemyoyclpiad.onion
\item
  Address is hard to remember.
\item
  HTTPS connection will fail since the certificate is not valid for the
  .onion addr.
\end{itemize}
\end{frame}

\begin{frame}[fragile]{Testing SNI}
\protect\hypertarget{testing-sni}{}
If we use OpenSSL via Tor, we can get the cert via Onion Service:

\begin{verbatim}
torsocks openssl s_client -servername autodefesa.org -tlsextdebug -connect \
  autodefcecpx2mut5medmyjxjg2wb6lwkbt3enl74frthemyoyclpiad.onion:443
\end{verbatim}
\end{frame}

\begin{frame}[fragile]{Using curl}
\protect\hypertarget{using-curl}{}
This could work in theory to fetch the site via Onion Services:

\begin{verbatim}
torsocks curl -vik --resolve \
  autodefesa.org:443:autodefcecpx2mut5medmyjxjg2wb6lwkbt3enl74frthemyoyclpiad.onion \
  https://autodefesa.org
\end{verbatim}

But won't work, since \texttt{curl(1)}'s \texttt{-\/-resolve} requires
an IP address.
\end{frame}

\begin{frame}[fragile]{Using OpenSSL}
\protect\hypertarget{using-openssl}{}
\begin{verbatim}
echo -e "GET / HTTP/1.1\r\nHost:autodefesa.org\r\n\r\nConnection: Close\r\n\r\n" | \
  torsocks openssl s_client -quiet -servername autodefesa.org -connect \
  autodefcecpx2mut5medmyjxjg2wb6lwkbt3enl74frthemyoyclpiad.onion:443
\end{verbatim}
\end{frame}

\end{document}
